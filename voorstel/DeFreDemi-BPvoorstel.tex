%==============================================================================
% Sjabloon onderzoeksvoorstel bachproef
%==============================================================================
% Gebaseerd op document class `hogent-article'
% zie <https://github.com/HoGentTIN/latex-hogent-article>

% Voor een voorstel in het Engels: voeg de documentclass-optie [english] toe.
% Let op: kan enkel na toestemming van de bachelorproefcoördinator!
\documentclass{hogent-article}
% Invoegen bibliografiebestand
\usepackage[style=apa]{biblatex}
\addbibresource{voorstel.bib}

% Informatie over de opleiding, het vak en soort opdracht
\studyprogramme{Professionele bachelor toegepaste informatica}
\course{Bachelorproef}
\assignmenttype{Onderzoeksvoorstel}
% Voor een voorstel in het Engels, haal de volgende 3 regels uit commentaar
% \studyprogramme{Bachelor of applied information technology}
% \course{Bachelor thesis}
% \assignmenttype{Research proposal}

\academicyear{2024-2025} % TODO: pas het academiejaar aan

% TODO: Werktitel
\title{Versterking van toegangsbeheer: Vulnerability \\ assessment en technische beveiligingsimplementatie van de nurse-PC als schakel tussen het IT- en OT-netwerk aan boord van een schip.}

% TODO: Studentnaam en emailadres invullen
\author{Demi De Fré}
\email{demi.defre@student.hogent.be}

% TODO: Medestudent
% Gaat het om een bachelorproef in samenwerking met een student in een andere
% opleiding? Geef dan de naam en emailadres hier
% \author{Yasmine Alaoui (naam opleiding)}
% \email{yasmine.alaoui@student.hogent.be}

% TODO: Geef de co-promotor op
\supervisor[Co-promotor]{K. Ternoey (DEME-Group, \href{mailto:Ternoey.Kristof@deme-group.com}{Ternoey.Kristof@deme-group.com})}

% Binnen welke specialisatierichting uit 3TI situeert dit onderzoek zich?
% Kies uit deze lijst:
%
% - Mobile \& Enterprise development
% - AI \& Data Engineering
% - Functional \& Business Analysis
% - System \& Network Administrator
% - Mainframe Expert
% - Als het onderzoek niet past binnen een van deze domeinen specifieer je deze
%   zelf
%
\specialisation{System \& Network Administrator}
\keywords{Schip, netwerkbeveiliging, toegangsbeheer, nurse-PC, proof-of-concept, \\ IT-netwerk, OT-netwerk}

\begin{document}

\begin{abstract}
  Dit onderzoek is gericht op het versterken van de netwerkbeveiliging met focus op de nurse-PC aan boord van een schip. 
  De nurse-PC is een werkstation dat fungeert als schakel tussen het IT- en OT-netwerk. Daarnaast, wordt dit systeem vaak gebruikt voor niet-geautoriseerde 
  activiteiten waardoor Shadow IT de beveiliging en controle over de toegang kan ondermijnen en het risico op cyberdreigingen toeneemt. Elke vertraging kan 
  belangrijke activiteiten, zoals het vertrek van het schip of baggerwerkzaamheden, in gevaar brengen wat aanzienlijke kosten met zich meebrengt. Dit onderzoek 
  focust zich op een specifieke casus binnen DEME-Group, waarbij de nurse-PC onmisbaar is voor het ELEK-team, dat deze dagelijks gebruikt voor het oplossen van 
  technische problemen. Hieruit volgt de hoofdonderzoeksvraag: "Hoe kan het toegangsbeheer van de nurse-PC op een schip worden versterkt om de veiligheid van 
  zowel het IT- als OT-netwerk aan boord te garanderen?” De methodologie omvat interviews met IT- en OT-specialisten, een vulnerability assessment van de nurse-PC 
  op een schip en de technische implementatie van beveiligingsmaatregelen via een proof-of-concept. Het onderzoek verwacht de beveiliging van de nurse-PC te versterken 
  en het risico op onbevoegde toegang te minimaliseren om zo de operationele continuïteit van het schip te waarborgen. Dit kan DEME-Group helpen om de netwerkveiligheid 
  te verbeteren en financiële verliezen door storingen te voorkomen. Daarnaast, kan het ELEK-team efficiënter werken alsook kunnen de IT- en netwerkbeheerders 
  ongeautoriseerde toegang beter tegengaan. Tenslotte kunnen de scheepsoperators en leidinggevenden rekenen op een veilige en operationele werkplek, wat bijdraagt 
  aan meer efficiëntie en veiligheid aan boord.
\end{abstract}

\tableofcontents

% De hoofdtekst van het voorstel zit in een apart bestand, zodat het makkelijk
% kan opgenomen worden in de bijlagen van de bachelorproef zelf.
%---------- Inleiding ---------------------------------------------------------

% TODO: Is dit voorstel gebaseerd op een paper van Research Methods die je
% vorig jaar hebt ingediend? Heb je daarbij eventueel samengewerkt met een
% andere student?
% Zo ja, haal dan de tekst hieronder uit commentaar en pas aan.

%\paragraph{Opmerking}

% Dit voorstel is gebaseerd op het onderzoeksvoorstel dat werd geschreven in het
% kader van het vak Research Methods dat ik (vorig/dit) academiejaar heb
% uitgewerkt (met medesturent VOORNAAM NAAM als mede-auteur).
% 

\section{Inleiding}%
\label{sec:inleiding}
Deze bachelorproef richt zich op een specifiek probleem binnen DEME-Group, een internationaal bedrijf dat zich specialiseert in maritieme projecten, waaronder 
baggerwerkzaamheden, \\offshore-activiteiten en infrastructuurwerken. \\DEME-Group heeft meer dan 100 schepen die wereldwijd worden ingezet voor verschillende maritieme projecten. Het probleem 
bevindt zich echter binnen de netwerkstructuur op hun schepen. Deze bachelorproef zoomt in op de netwerkstructuur en nurse-PC op één bepaald schip. De netwerkstructuur bestaat uit twee netwerken: het IT-netwerk voor administratieve taken 
en het OT-netwerk voor de operationele systemen. Deze twee netwerken zijn met elkaar verbonden via de nurse-PC, 
die fungeert als schakel en toegangspunt tussen deze twee netwerken. De focus van deze bachelorproef ligt op het afbakenen en versterken van die schakel.

\subsection{Probleemstelling en hoofdonderzoeksvraag}
De ingenieurs van het ELEK-team aan boord moeten snel en efficiënt problemen via de nurse-PC oplossen wanneer zich storingen voordoen in het OT- of  IT-netwerk.  
Als het probleem niet snel wordt opgelost, kunnen belangrijke werkzaamheden, zoals bijvoorbeeld het vertrek van het schip of het baggerwerkzaambheden, niet doorgaan. 
Dit brengt aanzienlijke kosten met zich mee. De nurse-PC wordt echter vaak misbruikt voor niet-geautoriseerde activiteiten. Dit heet Shadow IT en leidt tot verlies van controle 
over de beveiliging en toegang tot de nurse-PC. Hierdoor nemen de risico’s voor beide netwerken aan boord toe. Het is daarom van cruciaal belang om de nurse-PC af 
te bakenen: het toegangsbeheer te optimaliseren en de beveiliging te versterken. Op basis van deze probleemstelling volgt de hoofdonderzoeksvraag: 
"Hoe kan het toegangsbeheer van de nurse-PC op een schip worden versterkt om de veiligheid van zowel het IT- als OT-netwerk aan boord te garanderen?” 

\subsection{Deelvragen}
- Hoe is het IT- en OT-netwerk ingericht in relatie tot de nurse-PC aan boord van een schip? \\     
- Op welke wijze wordt toegang verleend tot de nurse-PC zowel intern als extern? \\    
- Wat zijn de specifieke kwetsbaarheden van de nurse-PC op basis van het toegangsbeheer? \\       
- Welke beveiligingsmaatregelen moeten worden geïmplementeerd op basis van de bevindingen uit de vulnerability assessment van het toegangsbeheer? \\
- Welke methoden kunnen worden gebruikt om Shadow IT beter te detecteren en in kaart te brengen? \\
- Hoe kan Shadow IT bijdragen aan de verspreiding van malware of andere cyberdreigingen binnen het netwerk?\\
- Wat zijn de gevolgen van Shadow IT met betrekking tot de regelgeving aan boord?\\
- Op welke manier kunnen ongeautoriseerde gebruikers toegang krijgen tot de nurse-PC?\\

\subsection{Doelgroep}
Een goed functionerende nurse-PC en veilige netwerken zijn essentieel voor de continuïteit van alle operaties aan boord. 
Dit kan de primaire doelgroep DEME-Group, helpen om de netwerkbeveiliging te verbeteren en financiële verliezen door storingen te voorkomen.
Binnen het bedrijf worden drie specifieke groepen als doelgroepen beschouwd: het ELEK-team aan boord, de systeem- en netwerkbeheerders voor zowel het IT- als het OT-netwerk, de scheepsoperators en de leidinggevenden.
Het ELEK-team, bestaande uit verschillende ingenieurs met elk hun eigen specialisatie, is verantwoordelijk voor het onderhoud en de werking van de operationele systemen aan boord. Bij technische problemen dienen zij snel en efficiënt te troubleshooten met de nurse-PC.  
Een veilige en goed afgebakende nurse-PC is voor hen van cruciaal belang om zonder vertraging toegang te krijgen tot de juiste systemen, om storingen in de IT- en OT-netwerken snel te verhelpen zodat het schip operationeel blijft.
De systeem -en netwerkbeheerders moeten ervoor zorgen dat onbevoegden geen toegang krijgen en dat de systemen beschermd zijn tegen cyberbedreigingen. Dit zal bijdragen aan de algemene netwerkbeveiliging.
Scheepsoperators en hun leidinggevenden (o.a. de kapitein, eerste stuurman, enz.) zijn verantwoordelijk voor de operationele veiligheid en efficiëntie aan boord. Een veilige nurse-PC is voor hen van groot belang om storingen te voorkomen, 
zodat de schepen probleemloos kunnen opereren en de veiligheid van de bemanning gewaarborgd blijft.

\subsection{Onderzoeksdoelstelling}
De onderzoeksdoelstelling is het opstellen van een gedetailleerd rapport waarin de geïdentificeerde kwetsbaarheden en aanbevelingen, als resultaat van de vulnerability assesment, worden genoteerd.
Op basis van dit rapport wordt er een proof-of-concept gedemonstreerd met alle beveiligingsimplementaties.
Daarnaast wordt er ook documentatie bijgehouden van alle stappen van de technische implementatie.
Het doel van de documentatie is dat de basis van de proof-of-concept in de toekomst kan worden gerepliceerd en herbruikt op de gehele vloot van DEME-Group.


%---------- Stand van zaken ---------------------------------------------------

\section{Literatuurstudie}%
\label{sec:literatuurstudie}
\subsection{Introductie}
In de afgelopen jaren is de kwetsbaarheid van schepen voor cyberaanvallen in een verontrustend tempo toegenomen. Weston \textcite{Hecker2021}, ethical hacker bij het security bedrijf 
Mission Secure, benadrukt in een interview met Declan Bush de vaak over het hoofd geziene zwaktes in de maritieme beveiliging. Hij
legt uit hoe kwetsbaar veel schepen zijn, ondanks de vooruitgang in informatiebeveiliging. \textcite{Hecker2021} bevestigt hoe schijnbaar onschuldige apparaten, zoals draadloze toetsenborden, printers en 
zelfs gebruikershandleidingen, door aanvallers kunnen worden misbruikt. Dit illustreert de gevaren van shadow IT, waarbij bijvoorbeeld 
ongecontroleerde en niet-geautoriseerde apparaten aan de nurse-PC kunnen worden gekoppeld, wat een groot risico vormt voor de operationele technologie aan boord. 
Deze technologie is vaak het belangrijkste doelwit voor aanvallers omdat deze systemen verantwoordelijk zijn voor de werking van het schip.

\subsection{De rol van de nurse-PC}
De Nurse-PC speelt de belangrijkste rol in de brugfunctie tussen het IT- en OT-netwerk. Deze specifieke computer bevindt 
zich tussen beide netwerken en fungeert als toegangspunt en tussenpersoon die belangrijke informatie en documentatie kan ophalen over verschillende systemen.
Het systeem stelt de technici in staat om toegang te krijgen tot gedetailleerde gegevens over de 
machines, zodat ze snel kunnen troubleshooten en technische problemen kunnen oplossen. De Nurse-PC zorgt ervoor dat alle benodigde documentatie
snel beschikbaar is voor het onderhoudsteam. Zo kunnen zij de juiste acties ondernemen om storingen 
te verhelpen en de systemen optimaal te laten functioneren (K. Ternoey, persoonlijke communicatie, 4 november 2024).

\subsection{Wat is Shadow IT}
Naast cyberaanvallen kunnen er ook schadelijke situaties ontstaan door Shadow IT binnen de organisatie. Hoewel dit vaak onbedoeld gebeurt, vormt dit probleem een 
grotere bedreiging dan externe aanvallen. Shadow IT verwijst naar IT-middelen die binnen een organisatie worden gebruikt zonder goedkeuring of beheer door de IT-afdeling. 
Dit kan variëren van persoonlijke apparaten tot cloudtechnologieën. Dergelijke technologieën zijn vaak niet in overeenstemming met de interne IT-processen en beveiligingsmaatregelen, waardoor ze een 
risico vormen voor de veiligheid van gegevens en systemen. Shadow IT kan leiden tot datadiefstal, 
verspreiding van malware of andere beveiligingsincidenten, omdat deze apparaten of diensten vaak niet voldoen aan de vereiste beveiligingsnormen. Hoewel het meestal niet opzettelijk is, 
ontstaat Shadow IT vaak doordat medewerkers officiële tools of processen als onvoldoende beschouwen om hun werk effectief te kunnen uitvoeren. Het is van essentieel belang dat organisaties 
deze praktijken herkennen en beheren om de bijbehorende risico’s te beperken \autocite{NCSC2023}.

\subsection{Shadow IT in de context van de Nurse-PC}
In de context van de nurse-PC aan boord van een schip vormt Shadow IT een ernstig probleem. Wanneer medewerkers persoonlijke apparaten of 
niet-goedgekeurde software gebruiken om toegang te krijgen tot systemen of gegevens, kunnen er ernstige beveiligingsproblemen ontstaan. 
Bijvoorbeeld, als de nurse-PC wordt gekoppeld aan onveilige hardware of software, kunnen gevoelige gegevens kwetsbaar worden voor datadiefstal of aanvallen, zoals eerder besproken in het interview met \textcite{Hecker2021}. 
Bovendien kunnen werknemers die betrokken zijn bij Shadow IT malware binnenhalen via niet-geautoriseerde apparaten of software, die zich via de nurse-PC kunnen verspreiden naar zowel 
het IT- als het OT-netwerk. Dit vergroot de kans op systeeminfecties, waardoor de integriteit van de machinebesturingssystemen ernstig in gevaar kan komen. (K. Ternoey, persoonlijke communicatie, 4 november 2024).

\subsection{Het verschil tussen een IT-netwerk en een OT-netwerk}
IT (informatietechnologie) en OT (operationele technologie) vervullen verschillende functies binnen een organisatie. IT richt zich op de verwerking, opslag en 
uitwisseling van gegevens, waarbij computers, servers en andere apparaten zoals smartphones en tablets worden gebruikt. OT daarentegen houdt zich bezig met het 
monitoren en besturen van fysieke processen en apparatuur, zoals industriële machines en besturingssystemen. Hoewel deze twee netwerken lange tijd gescheiden waren, 
wordt de scheidslijn steeds vager door de opkomst van IT/OT-convergentie, waarbij beide systemen steeds meer met elkaar worden geïntegreerd. Dit biedt nieuwe kansen, 
maar brengt ook risico’s met zich mee, vooral op het gebied van cybersecurity \autocite{onlogic2023}.

\subsection{IT/OT-convergentie}
De convergentie van Informatie Technologie (IT) en Operationele Technologie (OT) is een proces waarbij de twee voorheen gescheiden werelden worden geïntegreerd, mede aangedreven door digitale 
transformatie en technologische innovaties zoals Internet of Things (IoT) en big data-analyse. Deze integratie maakt een naadloze gegevensstroom mogelijk tussen de digitale en fysieke werelden, 
waardoor operationele systemen efficiënter kunnen worden aangestuurd en geoptimaliseerd. Er zijn drie belangrijke vormen van IT/OT-convergentie: fysieke convergentie (directe verbinding van OT-apparaten met IT-netwerken), 
softwareconvergentie \\(waarbij OT-gegevens digitaal worden geanalyseerd door IT-systemen), en organisatorische convergentie (waarbij de werkstromen van IT en OT samensmelten om de samenwerking te verbeteren). Deze integratie 
leidt tot betere besluitvorming, verhoogde efficiëntie en innovatie, wat cruciaal is voor de vooruitgang van de Industrie 4.0 en het Industrial Internet of Things (IIoT) \autocite{maleh2021ot,paloaltonetworks2023}.

\subsection{OT-security}
OT-security richt zich op de bescherming van operationele technologieën die worden gebruikt in industriële netwerken. Naarmate de connectiviteit van deze systemen 
met externe netwerken toeneemt, nemen de risico’s van cyberaanvallen toe. OT-security omvat een breed scala aan maatregelen en technologieën die ontworpen zijn om 
de betrouwbaarheid en veiligheid van industriële systemen te waarborgen, zoals SCADA (Supervisory Control And Data Acquisition) en Industrial Control Systems (ICS). Deze systemen worden vaak ingezet in vitale 
sectoren zoals energie, transport en de scheepvaart. De integratie van IT- en OT-netwerken, bekend als IT/OT-convergentie, biedt meer efficiëntie, maar brengt ook nieuwe 
beveiligingsuitdagingen met zich mee. Organisaties moeten daarom een specifieke \\OT-securitystrategie ontwikkelen die gericht is op het beschermen van deze cruciale systemen 
zonder de operationele processen te verstoren. Dit vereist intensieve monitoring en analyse van het netwerkverkeer om afwijkingen te detecteren die kunnen wijzen op aanvallen. 
Een snelle reactie op dreigingen is essentieel, omdat aanvallen op OT-systemen ernstige gevolgen kunnen hebben voor zowel de veiligheid van de infrastructuur als de bredere economie \autocite{Nomios2024}.


% Voor literatuurverwijzingen zijn er twee belangrijke commando's:
% \autocite{KEY} => (Auteur, jaartal) Gebruik dit als de naam van de auteur
%   geen onderdeel is van de zin.
% \textcite{KEY} => Auteur (jaartal)  Gebruik dit als de auteursnaam wel een
%   functie heeft in de zin (bv. ``Uit onderzoek door Doll & Hill (1954) bleek
%   ...'')

%---------- Methodologie ------------------------------------------------------
\section{Methodologie}%
\label{sec:methodologie}
In dit onderzoek worden allereerst interviews afgenomen met interne IT- en OT-specialisten om de nodige kennis te verkrijgen over de netwerkstructuur en de nurse-PC.
Hierdoor, kan er een goed inzicht gecreëerd worden in de huidige situatie. De vragen en antwoorden worden opgenomen in een document.
Vervolgens, wordt er een vulnerability assessment uitgevoerd om de toegang tot de nurse-PC te evalueren en mogelijke relevante kwetsbaarheden in het systeem of netwerkstructuur te identificeren. 
Dit resulteert in een rapport met de bevindingen van de vulnerability assessment.
Op basis van de resultaten van de assessment, worden de benodigde beveiligingsmaatregelen geïmplementeerd en getest door middel van een proof-of-concept. Dit gebeurt in een labo bij DEME-Group dat de netwerkstructuur van een schip volledig simuleert.  
De systemen beschikken over een Windows Operating System. Voor configuratie en automatisering van het toegangsbeheer en beveiligingsimplementatie, wordt er gewerkt met Powershell. 
Als extra hulpmiddel, zal er op een fysieke laptop, een virtuele machine aangemaakt worden met Vagrant die een Windows 10 desktop omgeving nabootst.
Tijdens de implementatie worden alle stappen, automatiseringsscripts, commando's en andere relevante details gedocumenteerd. Het doel is om in de toekomst, de proof-of-concept op een zowel reproduceerbare, repliceerbare en herbruikbare manier uit te voeren.


%---------- Verwachte resultaten ----------------------------------------------
\section{Verwacht resultaat}%
\label{sec:verwachte_resultaten}
Het onderzoek zal naar verwachting leiden tot een versterking van het toegangsbeheer en de beveiliging van de nurse-PC aan boord van een schip.
Bovendien, zorgt de gedocumenteerde proof-of-concept ervoor dat de geïmplementeerde beveiligingsmaatregelen in de toekomst eenvoudig gerepliceerd en herbruikt kunnen worden op andere schepen binnen de vloot.
De meerwaarde voor DEME-Group van deze bachelorproef, ligt in de praktische en technische toepasbaarheid en implementatie van de proof-of-concept. 
De verbeterde beveiliging draagt bij aan de operationele continuïteit en efficiëntie van schepen en biedt waardevolle inzichten voor het integreren van cybersecurity in industriële netwerken. 
De implementatie van deze maatregelen draagt ook bij aan een duurzamer en effectiever beheer van de netwerken, wat op zich de operationele kosten verlaagt.
Tenslotte kunnen de scheepsoperators en leidinggevenden van DEME-Group rekenen op een veilige en operationele werkplek, wat bijdraagt aan meer efficiëntie en veiligheid aan boord.


\section{Referentielijst}%
\label{sec:Referentielijst}


\printbibliography
\end{document}