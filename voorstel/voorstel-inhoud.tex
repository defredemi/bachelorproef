%---------- Inleiding ---------------------------------------------------------

% TODO: Is dit voorstel gebaseerd op een paper van Research Methods die je
% vorig jaar hebt ingediend? Heb je daarbij eventueel samengewerkt met een
% andere student?
% Zo ja, haal dan de tekst hieronder uit commentaar en pas aan.

%\paragraph{Opmerking}

% Dit voorstel is gebaseerd op het onderzoeksvoorstel dat werd geschreven in het
% kader van het vak Research Methods dat ik (vorig/dit) academiejaar heb
% uitgewerkt (met medesturent VOORNAAM NAAM als mede-auteur).
% 

\section{Inleiding}%
\label{sec:inleiding}
Deze bachelorproef richt zich op een specifiek probleem binnen DEME-Group, een internationaal bedrijf dat zich specialiseert in maritieme projecten, waaronder 
baggerwerkzaamheden, \\offshore-activiteiten en infrastructuurwerken. \\DEME-Group heeft meer dan 100 schepen die wereldwijd worden ingezet voor verschillende maritieme projecten. Het probleem 
bevindt zich echter binnen de netwerkstructuur op hun schepen. Deze bachelorproef zoomt in op de netwerkstructuur en nurse-PC op één bepaald schip. De netwerkstructuur bestaat uit twee netwerken: het IT-netwerk voor administratieve taken 
en het OT-netwerk voor de operationele systemen. Deze twee netwerken zijn met elkaar verbonden via de nurse-PC, 
die fungeert als schakel en toegangspunt tussen deze twee netwerken. De focus van deze bachelorproef ligt op het afbakenen en versterken van die schakel.

\subsection{Probleemstelling en hoofdonderzoeksvraag}
De ingenieurs van het ELEK-team aan boord moeten snel en efficiënt problemen via de nurse-PC oplossen wanneer zich storingen voordoen in het OT- of  IT-netwerk.  
Als het probleem niet snel wordt opgelost, kunnen belangrijke werkzaamheden, zoals bijvoorbeeld het vertrek van het schip of het baggerwerkzaambheden, niet doorgaan. 
Dit brengt aanzienlijke kosten met zich mee. De nurse-PC wordt echter vaak misbruikt voor niet-geautoriseerde activiteiten. Dit heet Shadow IT en leidt tot verlies van controle 
over de beveiliging en toegang tot de nurse-PC. Hierdoor nemen de risico’s voor beide netwerken aan boord toe. Het is daarom van cruciaal belang om de nurse-PC af 
te bakenen: het toegangsbeheer te optimaliseren en de beveiliging te versterken. Op basis van deze probleemstelling volgt de hoofdonderzoeksvraag: 
"Hoe kan het toegangsbeheer van de nurse-PC op een schip worden versterkt om de veiligheid van zowel het IT- als OT-netwerk aan boord te garanderen?” 

\subsection{Deelvragen}   
- Wat zijn de specifieke kwetsbaarheden van de nurse-PC op basis van het toegangsbeheer? \\       
- Welke beveiligingsmaatregelen moeten worden geïmplementeerd op basis van de bevindingen uit de vulnerability assessment van het toegangsbeheer? \\
- Welke methoden kunnen worden gebruikt om Shadow IT beter te detecteren en in kaart te brengen? \\
- Hoe kan Shadow IT bijdragen aan de verspreiding van malware of andere cyberdreigingen binnen het netwerk?\\
- Op welke wijze wordt toegang verleend tot de nurse-PC zowel intern als extern? \\

\subsection{Doelgroep}
Een goed functionerende nurse-PC en veilige netwerken zijn essentieel voor de continuïteit van alle operaties aan boord. 
Dit kan de primaire doelgroep DEME-Group, helpen om de netwerkbeveiliging te verbeteren en financiële verliezen door storingen te voorkomen.
Binnen het bedrijf worden drie specifieke groepen als doelgroepen beschouwd: het ELEK-team aan boord, de systeem- en netwerkbeheerders voor zowel het IT- als het OT-netwerk, de scheepsoperators en de leidinggevenden.
Het ELEK-team, bestaande uit verschillende ingenieurs met elk hun eigen specialisatie, is verantwoordelijk voor het onderhoud en de werking van de operationele systemen aan boord. Bij technische problemen dienen zij snel en efficiënt te troubleshooten met de nurse-PC.  
Een veilige en goed afgebakende nurse-PC is voor hen van cruciaal belang om zonder vertraging toegang te krijgen tot de juiste systemen, om storingen in de IT- en OT-netwerken snel te verhelpen zodat het schip operationeel blijft.
De systeem -en netwerkbeheerders moeten ervoor zorgen dat onbevoegden geen toegang krijgen en dat de systemen beschermd zijn tegen cyberbedreigingen. Dit zal bijdragen aan de algemene netwerkbeveiliging.
Scheepsoperators en hun leidinggevenden (o.a. de kapitein, eerste stuurman, enz.) zijn verantwoordelijk voor de operationele veiligheid en efficiëntie aan boord. Een veilige nurse-PC is voor hen van groot belang om storingen te voorkomen, 
zodat de schepen probleemloos kunnen opereren en de veiligheid van de bemanning gewaarborgd blijft.

\subsection{Onderzoeksdoelstelling}
De onderzoeksdoelstelling is het opstellen van een gedetailleerd rapport waarin de geïdentificeerde kwetsbaarheden en aanbevelingen, als resultaat van de vulnerability assesment, worden genoteerd.
Op basis van dit rapport wordt er een proof-of-concept gedemonstreerd met alle beveiligingsimplementaties.
Daarnaast wordt er ook documentatie bijgehouden van alle stappen van de technische implementatie.
Het doel van de documentatie is dat de basis van de proof-of-concept in de toekomst kan worden gerepliceerd en herbruikt op de gehele vloot van DEME-Group.


%---------- Stand van zaken ---------------------------------------------------

\section{Literatuurstudie}%
\label{sec:literatuurstudie}
\subsection{Introductie}
In de afgelopen jaren, is de kwetsbaarheid van schepen door cyberaanvallen, in een verontrustend tempo toegenomen. Weston \textcite{Hecker2021}, ethical hacker bij het security bedrijf 
Mission Secure, benadrukt in een interview met Declan Bush de zwaktes die vaak over het hoofd gezien worden in de maritieme beveiliging. Hij
legt uit hoe kwetsbaar veel schepen zijn ondanks de vooruitgang in de informatiebeveiliging. \textcite{Hecker2021} bevestigt hoe schijnbaar onschuldige apparaten zoals draadloze toetsenborden, printers en 
zelfs gebruikershandleidingen door aanvallers worden misbruikt. Dit illustreert de gevaren van Shadow IT waarbij ongecontroleerde en niet-geautoriseerde apparaten toegang kunnen krijgen tot kritieke systemen aan boord. 
In het bijzonder vormt het koppelen van dergelijke apparaten aan de nurse-PC een groot risico. De nurse-PC fungeert als een brug tussen het IT- en OT-netwerk en vormt daarmee een belangrijk toegangspunt. 
Aangezien de operationele technologie aan boord vaak het primaire doelwit is, is het belangrijk om de beveiligingsrisico's van Shadow IT in de context van de nurse-PC grondig te onderzoeken.
Deze literatuurstudie gaat dieper in op de rol van de nurse-PC, Shadow IT, IT/OT-convergentie en OT-security.

\subsection{De rol van de nurse-PC} 
De nurse-PC vervult de belangrijkste rol in de brugfunctie tussen het IT- en OT-netwerk. Deze specifieke computer bevindt 
zich tussen beide netwerken en fungeert als schakel en als toegangspunt. De nurse-PC haalt informatie en documentatie op over verschillende systemen heen.
Het ELEK-team kan dan toegang krijgen tot gedetailleerde gegevens over de machines. 
Zo kan het team snel troubleshooten en technische problemen oplossen. 
De nurse-PC maakt alle benodigde documentatie snel toegankelijk zodat storingen worden verholpen en de systemen optimaal blijven functioneren. (K. Ternoey, persoonlijke communicatie, 4 november 2024).

\subsection{Shadow IT}
De kwetsbaarheid van de nurse-PC neemt toe door de oncontroleerbare invloed van Shadow IT.
Shadow IT verwijst naar IT-middelen die binnen een organisatie worden gebruikt zonder goedkeuring of beheer door de IT-afdeling.
Het gebruik van persoonlijke apparaten en cloudtechnologieën stemt vaak niet overeen met de interne IT-processen en beveiligingsmaatregelen waardoor ze een 
risico vormen voor de veiligheid van gegevens en systemen. Shadow IT kan leiden tot diefstal van data, 
verspreiding van malware of andere beveiligingsincidenten omdat deze apparaten of diensten vaak niet voldoen aan de vereiste beveiligingsnormen. 
Het is van essentieel belang dat organisaties deze praktijken herkennen en beheren om de bijbehorende risico’s te beperken \autocite{NCSC2023}.
Wanneer medewerkers persoonlijke apparaten of niet-goedgekeurde software gebruiken om toegang te krijgen tot systemen of gegevens, kunnen er beveiligingsproblemen 
ontstaan. In het geval dat de nurse-PC wordt gekoppeld aan onveilige hardware of software, worden gevoelige gegevens kwetsbaar voor 
datadiefstal of aanvallen zoals eerder besproken in het interview met \textcite{Hecker2021}.
Bovendien, kunnen werknemers die betrokken zijn bij Shadow IT, malware binnenhalen via niet-geautoriseerde apparaten of software. 
De malware kan zich dan via de nurse-PC verspreiden naar zowel het IT- als het OT-netwerk. 
Dit vergroot de kans op systeeminfecties. De integriteit van de machinebesturingssystemen loopt hierdoor ernstig in
gevaar (K. Ternoey, persoonlijke communicatie, 4 november 2024).

\subsection{IT/OT-convergentie}
Shadow IT vormt een bedreiging voor de beveiliging. 
Tegelijkertijd biedt de IT/OT-convergentie nieuwe mogelijkheden voor efficiëntie maar introduceert het ook extra kwetsbaarheden die zorgvuldige aandacht vereisen, 
vooral met betrekking tot de nurse-PC.
Informatie Technologie (IT) en Operationele \\Technologie (OT) vervullen verschillende functies binnen een organisatie. IT richt zich op de verwerking, opslag en 
uitwisseling van gegevens. De computers, servers en andere apparaten zoals \\smartphones en tablets maken hier gebruik van. Daarentegen, houdt OT zich bezig met het 
monitoren en besturen van fysieke processen en apparatuur zoals industriële machines. Hoewel deze twee netwerken lange tijd gescheiden waren, 
\\wordt de scheidingslijn steeds vager. Door de opkomst van IT/OT-convergentie worden beide systemen steeds meer met elkaar geïntegreerd. Dit biedt nieuwe kansen
maar brengt ook risico’s met zich mee, vooral op het gebied van cybersecurity. \autocite{onlogic2023}.
De convergentie van Informatie Technologie (IT) en Operationele Technologie (OT) is een proces waarbij beide werelden worden geïntegreerd. 
Dit proces wordt mede aangedreven door digitale transformatie en technologische innovaties zoals Internet of Things (IoT) en Big data-analyse. 
Deze integratie maakt een vlotte gegevensstroom mogelijk tussen de digitale en fysieke werelden
waardoor operationele systemen efficiënter worden aangestuurd en geoptimaliseerd. 
De integratie van IT- en OT-netwerken maakt de nurse-PC een cruciale schakel tussen operationele systemen en informatiesystemen aan boord, 
wat de efficiëntie van het schip verbetert. Door real-time gegevens en storingsanalyses te leveren, vergemakkelijkt de nurse-PC snel herstel van technische problemen. 
Echter, deze verbondenheid maakt de nurse-PC kwetsbaar voor cyberaanvallen omdat ongecontroleerde apparaten via Shadow IT gemakkelijker toegang kunnen krijgen. 
De complexiteit van de IT/OT-convergentie verhoogt de beveiligingsuitdagingen omdat de nurse-PC momenteel een zwakke schakel vormt door Shadow IT.
Er zijn drie belangrijke vormen van IT/OT-convergentie: fysieke convergentie, software convergentie en organisatorische convergentie. 
Fysieke convergentie is een directe verbinding van OT-apparaten met IT-netwerken. Software convergentie is
de digitale analyse van OT-gegevens door IT-systemen. De organisatorische convergentie is de samensmelting van de IT- en OT-werkstromen om de samenwerking te verbeteren.
Deze integratie leidt tot betere besluitvorming, verhoogde efficiëntie en innovatie wat belangrijk is voor de vooruitgang van de Industrie 4.0. \autocite{maleh2021ot,paloaltonetworks2023}.

\subsection{OT-security}
Hoewel IT/OT-convergentie nieuwe mogelijkheden biedt, brengt het ook nieuwe beveiligingsuitdagingen met zich mee zoals de bescherming van operationele systemen.
OT-security richt zich op de bescherming van operationele technologieën die worden gebruikt in industriële netwerken. Naarmate de connectiviteit van deze systemen 
met externe netwerken toeneemt, nemen de risico’s van cyberaanvallen toe. OT-security omvat een breed scala aan maatregelen en technologieën die ontworpen zijn om 
de betrouwbaarheid en veiligheid van industriële systemen zoals SCADA (Supervisory Control And Data Acquisition) en Industrial Control Systems (ICS) te waarborgen. Deze systemen worden vaak ingezet in vitale 
sectoren zoals energie, transport en scheepvaart. IT/OT-convergentie biedt meer efficiëntie maar brengt ook nieuwe 
beveiligingsuitdagingen met zich mee. Organisaties dienen daarom een specifieke OT security strategie te ontwikkelen die gericht is op het beschermen van cruciale systemen 
zonder de operationele processen te verstoren. Dit vereist intensieve monitoring en analyse van het netwerkverkeer om afwijkingen te detecteren. 
Een snelle reactie op dreigingen is essentieel omdat aanvallen op OT-systemen ernstige gevolgen hebben voor de infrastructuur. \autocite{Nomios2024}.

\subsection{Vulnerability assesment}
Vooraleer de technische uitdagingen van de nurse-PC worden aangepakt, moet een vulnerability assessment worden uitgevoerd. 
Het proces begint met de voorbereiding, waarbij de scope van het assessment wordt vastgesteld. Dit houdt in dat alle hardware en 
systemen die verbonden zijn met de nurse-PC in kaart worden gebracht, met speciale aandacht voor ongeautoriseerde apparaten die 
via Shadow IT toegang proberen te krijgen. Vervolgens worden geautomatiseerde tests uitgevoerd, zoals netwerk- en applicatiescans, 
om kwetsbaarheden te identificeren. Hierbij wordt gekeken naar zwakke plekken in het toegangsbeheer, zoals onvoldoende beveiligde 
toegangspunten of onjuiste toegangsrechten voor gebruikers en apparaten. Na de identificatie van deze kwetsbaarheden, worden ze 
geanalyseerd op basis van hun ernst en de potentiële impact op zowel de nurse-PC als de gekoppelde systemen. De kwetsbaarheden worden 
geprioriteerd, waarbij vooral de meest risicovolle factoren, zoals onbevoegde toegang of kwetsbaarheden in de netwerkverbindingen, 
naar voren komen. Het eindresultaat is een gedetailleerd rapport met bevindingen en aanbevelingen voor het versterken van de beveiliging, 
zoals het implementeren van strengere toegangsbeperkingen, het blokkeren van ongeautoriseerde apparaten en het verbeteren van de algemene 
netwerkbeveiliging. \autocite{HON2024}.

\subsection{Conclusie literatuuronderzoek}
Uit de literatuurstudie kan er effectief worden geconcludeerd dat de nurse-PC een cruciale schakel vormt tussen IT- en OT-netwerken maar ook een kwetsbaar doelwit is voor cyberaanvallen. 
Shadow IT vormt een extra risico door ongeautoriseerde toegang en malware terwijl IT/OT-\\convergentie nieuwe uitdagingen creëert voor de beveiliging. 
Deze bevindingen benadrukken de behoefte aan verder onderzoek. In de volgende sectie wordt de methodologie besproken waarin de stappen van het onderzoek 
en de technische implementatie worden toegelicht. Het doel is het versterken van het toegangsbeheer van de nurse-PC zodat de veiligheid van zowel het IT- als OT-netwerk aan boord 
effectief wordt gewaarborgd en Shadow IT kan worden voorkomen.


% Voor literatuurverwijzingen zijn er twee belangrijke commando's:
% \autocite{KEY} => (Auteur, jaartal) Gebruik dit als de naam van de auteur
%   geen onderdeel is van de zin.
% \textcite{KEY} => Auteur (jaartal)  Gebruik dit als de auteursnaam wel een
%   functie heeft in de zin (bv. ``Uit onderzoek door Doll & Hill (1954) bleek
%   ...'')

%---------- Methodologie ------------------------------------------------------
\section{Methodologie}%
\label{sec:methodologie}
In dit onderzoek worden allereerst interviews afgenomen met interne IT- en OT-specialisten om de nodige kennis te verkrijgen over de netwerkstructuur en de nurse-PC.
Hierdoor, kan er een goed inzicht gecreëerd worden in de huidige situatie. De vragen en antwoorden worden opgenomen in een document.
Vervolgens, wordt er een vulnerability assessment uitgevoerd om de toegang tot de nurse-PC te evalueren en mogelijke relevante kwetsbaarheden in het systeem of netwerkstructuur te identificeren. 
Dit resulteert in een rapport met de bevindingen van de vulnerability assessment.
Op basis van de resultaten van de assessment, worden de benodigde beveiligingsmaatregelen geïmplementeerd en getest door middel van een proof-of-concept. Dit gebeurt in een labo bij DEME-Group dat de netwerkstructuur van een schip volledig simuleert.  
De systemen beschikken over een Windows Operating System. Voor configuratie en automatisering van het toegangsbeheer en beveiligingsimplementatie, wordt er gewerkt met Powershell. 
Als extra hulpmiddel, zal er op een fysieke laptop, een virtuele machine aangemaakt worden met Vagrant die een Windows 10 desktop omgeving nabootst.
Tijdens de implementatie worden alle stappen, automatiseringsscripts, commando's en andere relevante details gedocumenteerd. Het doel is om in de toekomst, de proof-of-concept op een zowel reproduceerbare, repliceerbare en herbruikbare manier uit te voeren.


%---------- Verwachte resultaten ----------------------------------------------
\section{Verwacht resultaat}%
\label{sec:verwachte_resultaten}
Het onderzoek zal naar verwachting leiden tot een versterking van het toegangsbeheer en de beveiliging van de nurse-PC aan boord van een schip.
Bovendien, zorgt de gedocumenteerde proof-of-concept ervoor dat de geïmplementeerde beveiligingsmaatregelen in de toekomst eenvoudig gerepliceerd en herbruikt kunnen worden op andere schepen binnen de vloot.
De meerwaarde voor DEME-Group van deze bachelorproef, ligt in de praktische en technische toepasbaarheid en implementatie van de proof-of-concept. 
De verbeterde beveiliging draagt bij aan de operationele continuïteit en efficiëntie van schepen en biedt waardevolle inzichten voor het integreren van cybersecurity in industriële netwerken. 
De implementatie van deze maatregelen draagt ook bij aan een duurzamer en effectiever beheer van de netwerken, wat op zich de operationele kosten verlaagt.
Tenslotte kunnen de scheepsoperators en leidinggevenden van DEME-Group rekenen op een veilige en operationele werkplek, wat bijdraagt aan meer efficiëntie en veiligheid aan boord.


\section{Referentielijst}%
\label{sec:Referentielijst}

