%---------- Inleiding ---------------------------------------------------------

% TODO: Is dit voorstel gebaseerd op een paper van Research Methods die je
% vorig jaar hebt ingediend? Heb je daarbij eventueel samengewerkt met een
% andere student?
% Zo ja, haal dan de tekst hieronder uit commentaar en pas aan.

%\paragraph{Opmerking}

% Dit voorstel is gebaseerd op het onderzoeksvoorstel dat werd geschreven in het
% kader van het vak Research Methods dat ik (vorig/dit) academiejaar heb
% uitgewerkt (met medesturent VOORNAAM NAAM als mede-auteur).
% 

\section{Inleiding}%
\label{sec:inleiding}
Deze bachelorproef richt zich op een concreet probleem binnen DEME-Group. Een internationaal bedrijf dat zich specialiseert in maritieme projecten zoals baggeren, offshore-activiteiten en infrastructuurwerken. 
DEME is actief in het ontwerpen, bouwen en onderhouden van complexe civiele en waterbouwkundige systemen. Ze werken met meer dan 100 schepen die wereldwijd worden ingezet voor verschillende maritieme projecten.
Het probleem bevindt zich echter binnen de netwerkstructuur van een schip, die bestaat uit twee netwerken: 
het IT-netwerk voor administratieve taken en het OT-netwerk voor de operationele systemen zoals de besturing van het schip. Deze twee netwerken zijn met elkaar verbonden via de nurse pc, die fungeert 
als toegangspunt tussen de twee netwerken. De focus van deze bachelorproef ligt op het afbakenen van de toegang tot deze kritieke schakel.

\subsection{Probleemstelling en hoofdonderzoeksvraag}
De ingenieurs van het ELEK-team (elektroteam) aan boord moeten snel en efficiënt problemen via de nurse pc kunnen oplossen wanneer zich storingen voordoen in het OT- of IT-netwerk. Als het probleem niet snel wordt opgelost, kunnen belangrijke werkzaamheden, 
zoals het vertrek van het schip of het baggeren, niet doorgaan. Dit brengt aanzienlijke kosten met zich mee.
De nurse pc wordt echter vaak misbruikt voor niet-geautoriseerde activiteiten (shadow IT). Dit leidt tot verlies van controle over de beveiliging en toegang tot de nurse pc. 
Hierdoor nemen de risico’s voor de twee netwerken aan boord toe. 
Het is daarom cruciaal om de nurse pc af te bakenen en te beveiligen. De toegangsbeheer moet versterkt worden.

Op basis van deze probleemstelling luidt de hoofdonderzoeksvraag:
"Hoe kan het toegangsbeheer van de nurse pc op een schip worden versterkt door een vulnerability assessment uit te voeren en vervolgens de juiste beveiligingsmaatregelen te implementeren om de veiligheid van zowel het IT- als OT-netwerk aan boord te garanderen?"

\subsection{Deelvragen}
- Hoe is de netwerkinfrastructuur aan boord van een schip ingericht in relatie tot de nurse pc?
- Op welke manieren wordt toegang verleend tot de nurse pc, zowel intern als extern?
- Wat zijn de specifieke kwetsbaarheden van de nurse pc op basis van toegangsbeheer?
- Hoe wordt de toegang tot de nurse pc gelogd?
- Welke beveiligingsmaatregelen moeten worden geïmplementeerd op basis van de bevindingen uit de vulnerability assessment van het toegangsbeheer?


\subsection{Doelgroep}
De hoofdgroep van de doelgroep is het bedrijf DEME-group, aangezien een goed functionerende nurse pc en veilige netwerken 
essentieel zijn voor de continuïteit van kritieke operaties aan boord. Het oplossen van beveiligingsproblemen voorkomt financiële verliezen en verhoogt de efficiëntie van de schepen.
Binnen het bedrijf zijn drie specifieke sectoren die als doelgroepen worden beschouwd: het ELEK-team aan boord bestaande uit meerdere ingenieurs elk met hun specialiteit, de IT- en netwerkbeheerders, en de scheepsoperators en leidinggevenden.
Het ELEK-team is is verantwoordelijk voor het onderhoud en de werking van de systemen aan boord. Wanneer er zich technische problemen voordoen, moeten zij snel en efficiënt kunnen troubleshooten met de nurse pc.
Een veilige nurse PC is essentieel voor hen om zonder vertraging toegang te krijgen tot de juiste systemen, zodat storingen in de IT- en OT-netwerken snel opgelost kunnen worden en het schip operationeel blijft.
De IT- en netwerkbeheerders moeten ervoor zorgen dat geen onbevoegden toegang krijgen en dat de systemen beschermd zijn tegen cyberdreigingen, wat de algehele netwerkveiligheid verbetert.
Scheepsoperators, kapiteins en leidinggevenden zijn verantwoordelijk voor de operationele veiligheid en efficiëntie aan boord. Een veilige nurse PC is cruciaal voor hen om storingen te voorkomen, zodat de schepen probleemloos kunnen opereren en de 
veiligheid van de bemanning en lading gewaarborgd blijft.


\subsection{Onderzoeksdoelstelling}
De onderzoeksdoelstelling is het opstellen van een gedetailleerd rapport waarin de geïdentificeerde kwetsbaarheden en aanbevelingen, als resultaat van de vulnerability assesment, worden genoteerd.
Op basis van dit rapport wordt er een proof-of-concept gedemonstreert met alle beveiligingsimplementaties.
Daarnaast wordt er ook documentatie bijgehouden van alle stappen om de beveiliging correct te implementeren.
Het doel van de documentatie is dat de basis van de proof-of-concept in de toekomst kan worden gerepliceerd op de gehele vloot.

%---------- Stand van zaken ---------------------------------------------------

\section{Literatuurstudie}%
\label{sec:literatuurstudie}
\subsection{Shadow-IT}
\subsection{Nurse pc}
\subsection{Active directory}
\subsection{Remote access}
\subsection{Monitoring}

% Voor literatuurverwijzingen zijn er twee belangrijke commando's:
% \autocite{KEY} => (Auteur, jaartal) Gebruik dit als de naam van de auteur
%   geen onderdeel is van de zin.
% \textcite{KEY} => Auteur (jaartal)  Gebruik dit als de auteursnaam wel een
%   functie heeft in de zin (bv. ``Uit onderzoek door Doll & Hill (1954) bleek
%   ...'')


%---------- Methodologie ------------------------------------------------------
\section{Methodologie}%
\label{sec:methodologie}

In dit onderzoek worden allereerst interviews afgenomen met interne IT-specialisten om de nodige kennis te verkrijgen over de netwerkstructuur en de nurse pc. 
Hierdoor kan er een goed inzicht gecreëerd worden in de huidige situatie. De vragen en antwoorden zullen opgenomen worden in een document.

Vervolgens wordt er een vulnerability assessment uitgevoerd om de toegang tot de nurse pc te evalueren en mogelijke kwetsbaarheden in het systeem 
of netwerkstructuur te identificeren. Hiervoor wordt er gewerkt met windows gebaseerde besturingsystemen.

Op basis van de resultaten van de assessment worden de benodigde beveiligingsmaatregelen geïmplementeerd en getest in een proof-of-concept. 
Dit gebeurt in een labo dat de netwerkstructuur van een schip nauwkeurig simuleert. Gedurende de implementatie worden alle stappen, commando's, etc. gedocumenteerd zodat 
de proof-of-concept repliceerbaar is.


%---------- Verwachte resultaten ----------------------------------------------
\section{Verwacht resultaat, conclusie}%
\label{sec:verwachte_resultaten}

Hier beschrijf je welke resultaten je verwacht. Als je metingen en simulaties uitvoert, kan je hier al mock-ups maken van de grafieken samen met de verwachte conclusies. Benoem zeker al je assen en de onderdelen van de grafiek die je gaat gebruiken. Dit zorgt ervoor dat je concreet weet welk soort data je moet verzamelen en hoe je die moet meten.

Wat heeft de doelgroep van je onderzoek aan het resultaat? Op welke manier zorgt jouw bachelorproef voor een meerwaarde?

Hier beschrijf je wat je verwacht uit je onderzoek, met de motivatie waarom. Het is \textbf{niet} erg indien uit je onderzoek andere resultaten en conclusies vloeien dan dat je hier beschrijft: het is dan juist interessant om te onderzoeken waarom jouw hypothesen niet overeenkomen met de resultaten.

