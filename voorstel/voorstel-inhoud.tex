%---------- Inleiding ---------------------------------------------------------

% TODO: Is dit voorstel gebaseerd op een paper van Research Methods die je
% vorig jaar hebt ingediend? Heb je daarbij eventueel samengewerkt met een
% andere student?
% Zo ja, haal dan de tekst hieronder uit commentaar en pas aan.

%\paragraph{Opmerking}

% Dit voorstel is gebaseerd op het onderzoeksvoorstel dat werd geschreven in het
% kader van het vak Research Methods dat ik (vorig/dit) academiejaar heb
% uitgewerkt (met medesturent VOORNAAM NAAM als mede-auteur).
% 

\section{Inleiding}%
\label{sec:inleiding}
Deze bachelorproef richt zich op een specifiek probleem binnen DEME-Group, een internationaal bedrijf dat zich specialiseert in maritieme projecten, waaronder 
baggerwerkzaamheden, \\offshore-activiteiten en infrastructuurwerken. \\DEME-Group heeft meer dan 100 schepen die wereldwijd worden ingezet voor verschillende maritieme projecten. Het probleem 
bevindt zich echter binnen de netwerkstructuur op hun schepen. Deze bachelorproef zoomt in op de netwerkstructuur en nurse-PC op één bepaald schip. De netwerkstructuur op het schip bestaat uit twee netwerken: het IT-netwerk voor administratieve taken 
en het OT-netwerk voor de operationele systemen zoals onder meer de besturing van het schip. Deze twee netwerken zijn met elkaar verbonden via de nurse-PC, 
die fungeert als schakel en toegangspunt tussen deze twee netwerken. De focus van deze bachelorproef ligt op het afbakenen en versterken van die schakel.

\subsection{Probleemstelling en hoofdonderzoeksvraag}
De ingenieurs van het ELEK-team aan boord moeten snel en efficiënt problemen via de nurse-PC oplossen wanneer zich storingen voordoen in het OT- of  IT-netwerk.  
Als het probleem niet snel wordt opgelost, kunnen belangrijke werkzaamheden, zoals bijvoorbeeld het vertrek van het schip of het baggerwerkzaambheden, niet doorgaan. 
Dit brengt aanzienlijke kosten met zich mee. De nurse-PC wordt echter vaak misbruikt voor niet-geautoriseerde activiteiten. Dit heet Shadow IT en leidt tot verlies van controle 
over de beveiliging en toegang tot de nurse-PC.  Hierdoor nemen de risico’s voor beide netwerken aan boord toe. Het is daarom van cruciaal belang om de nurse-PC af 
te bakenen: het toegangsbeheer te optimaliseren en de beveiliging te versterken. Op basis van deze probleemstelling volgt de hoofdonderzoeksvraag: 
"Hoe kan het toegangsbeheer van de nurse-PC op een schip worden versterkt om de veiligheid van zowel het IT- als OT-netwerk aan boord te garanderen?” 

\subsection{Deelvragen}
- Hoe is het IT- en OT-netwerk ingericht in relatie tot de nurse-PC aan boord van een schip? \\     
- Op welke wijze wordt toegang verleend tot de nurse-PC zowel intern als extern? \\    
- Wat zijn de specifieke kwetsbaarheden van de nurse-PC op basis van het toegangsbeheer? \\       
- Welke beveiligingsmaatregelen moeten worden geïmplementeerd op basis van de bevindingen uit de vulnerability assessment van het toegangsbeheer? \\
- Welke methoden kunnen worden gebruikt om Shadow IT beter te detecteren en in kaart te brengen? \\
- Hoe kan Shadow IT bijdragen aan de verspreiding van malware of andere cyberdreigingen binnen het netwerk?\\
- Wat zijn de gevolgen van Shadow IT met betrekking tot de regelgeving aan boord?\\
- Op welke manier kunnen ongeautoriseerde gebruikers toegang krijgen tot de nurse-PC?\\

\subsection{Doelgroep}
Een goed functionerende nurse-PC en veilige netwerken zijn essentieel voor de continuïteit van alle operaties aan boord. 
Dit kan de primaire doelgroep DEME-Group, helpen om de netwerkbeveiliging te verbeteren en financiële verliezen door storingen te voorkomen.
Binnen het bedrijf worden drie specifieke groepen als doelgroepen beschouwd: het ELEK-team aan boord, de systeem- en netwerkbeheerders voor zowel het IT- als het OT-netwerk, de scheepsoperators en de leidinggevenden.
Het ELEK-team, bestaande uit verschillende ingenieurs met elk hun eigen specialisatie, is verantwoordelijk voor het onderhoud en de werking van de operationele systemen aan boord. Bij technische problemen dienen zij snel en efficiënt te troubleshooten met de nurse-PC.  
Een veilige en goed afgebakende nurse-PC is voor hen van cruciaal belang om zonder vertraging toegang te krijgen tot de juiste systemen, om storingen in de IT- en OT-netwerken snel te verhelpen zodat het schip operationeel blijft.
De systeem -en netwerkbeheerders moeten ervoor zorgen dat onbevoegden geen toegang krijgen en dat de systemen beschermd zijn tegen cyberbedreigingen. Dit zal bijdragen aan de algemene netwerkbeveiliging.
Scheepsoperators en hun leidinggevenden (o.a. de kapitein, eerste stuurman, enz.) zijn verantwoordelijk voor de operationele veiligheid en efficiëntie aan boord. Een veilige nurse-PC is voor hen van groot belang om storingen te voorkomen, 
zodat de schepen probleemloos kunnen opereren en de veiligheid van de bemanning gewaarborgd blijft.

\subsection{Onderzoeksdoelstelling}
De onderzoeksdoelstelling is het opstellen van een gedetailleerd rapport waarin de geïdentificeerde kwetsbaarheden en aanbevelingen, als resultaat van de vulnerability assesment, worden genoteerd.
Op basis van dit rapport wordt er een proof-of-concept gedemonstreerd met alle beveiligingsimplementaties.
Daarnaast wordt er ook documentatie bijgehouden van alle stappen van de technische implementatie.
Het doel van de documentatie is dat de basis van de proof-of-concept in de toekomst kan worden gerepliceerd en herbruikt op de gehele vloot van DEME-Group.


%---------- Stand van zaken ---------------------------------------------------

\section{Literatuurstudie}%
\label{sec:literatuurstudie}
\subsection{IT vs.OT }
\subsection{Nurse-PC}
\subsection{Shadow IT}
\subsection{Koppeling tussen het IT-en OT-netwerk}


% Voor literatuurverwijzingen zijn er twee belangrijke commando's:
% \autocite{KEY} => (Auteur, jaartal) Gebruik dit als de naam van de auteur
%   geen onderdeel is van de zin.
% \textcite{KEY} => Auteur (jaartal)  Gebruik dit als de auteursnaam wel een
%   functie heeft in de zin (bv. ``Uit onderzoek door Doll & Hill (1954) bleek
%   ...'')

%---------- Methodologie ------------------------------------------------------
\section{Methodologie}%
\label{sec:methodologie}
In dit onderzoek worden allereerst interviews afgenomen met interne IT- en OT-specialisten om de nodige kennis te verkrijgen over de netwerkstructuur en de nurse-PC.
Hierdoor, kan er een goed inzicht gecreëerd worden in de huidige situatie. De vragen en antwoorden worden opgenomen in een document.
Vervolgens, wordt er een vulnerability assessment uitgevoerd om de toegang tot de nurse-PC te evalueren en mogelijke relevante kwetsbaarheden in het systeem of netwerkstructuur te identificeren. 
Dit resulteert in een rapport met de bevindingen van de vulnerability assessment.
Op basis van de resultaten van de assessment, worden de benodigde beveiligingsmaatregelen geïmplementeerd en getest door middel van een proof-of-concept. Dit gebeurt in een labo bij DEME-Group dat de netwerkstructuur van een schip volledig simuleert.  
De systemen beschikken over een Windows Operating System. Voor configuratie en automatisering van het toegangsbeheer en beveiligingsimplementatie, wordt er gewerkt met Powershell. 
Als extra hulpmiddel, zal er op een fysieke laptop, een virtuele machine aangemaakt worden met Vagrant die een Windows 10 desktop omgeving nabootst.
Tijdens de implementatie worden alle stappen, automatiseringsscripts, commando's en andere relevante details gedocumenteerd. Het doel is om in de toekomst, de proof-of-concept op een zowel reproduceerbare, repliceerbare en herbruikbare manier uit te voeren.


%---------- Verwachte resultaten ----------------------------------------------
\section{Verwacht resultaat}%
\label{sec:verwachte_resultaten}
Het onderzoek zal naar verwachting leiden tot een versterking van het toegangsbeheer en de beveiliging van de nurse-PC aan boord van een schip.
Bovendien, zorgt de gedocumenteerde proof-of-concept ervoor dat de geïmplementeerde beveiligingsmaatregelen in de toekomst eenvoudig gerepliceerd en herbruikt kunnen worden op andere schepen binnen de vloot.
De meerwaarde voor DEME-Group van deze bachelorproef, ligt in de praktische en technische toepasbaarheid en implementatie van de proof-of-concept. 
De verbeterde beveiliging draagt bij aan de operationele continuïteit en efficiëntie van schepen en biedt waardevolle inzichten voor het integreren van cybersecurity in industriële netwerken. 
De implementatie van deze maatregelen draagt ook bij aan een duurzamer en effectiever beheer van de netwerken, wat op zich de operationele kosten verlaagt.
Tenslotte kunnen de scheepsoperators en leidinggevenden van DEME-Group rekenen op een veilige en operationele werkplek, wat bijdraagt aan meer efficiëntie en veiligheid aan boord.


% \section{Referentielijst}%

