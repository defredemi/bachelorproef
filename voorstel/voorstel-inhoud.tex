%---------- Inleiding ---------------------------------------------------------

% TODO: Is dit voorstel gebaseerd op een paper van Research Methods die je
% vorig jaar hebt ingediend? Heb je daarbij eventueel samengewerkt met een
% andere student?
% Zo ja, haal dan de tekst hieronder uit commentaar en pas aan.

%\paragraph{Opmerking}

% Dit voorstel is gebaseerd op het onderzoeksvoorstel dat werd geschreven in het
% kader van het vak Research Methods dat ik (vorig/dit) academiejaar heb
% uitgewerkt (met medesturent VOORNAAM NAAM als mede-auteur).
% 

\section{Inleiding}%
\label{sec:inleiding}
Deze bachelorproef richt zich op een concreet probleem binnen DEME-Group. Een internationaal bedrijf dat zich specialiseert in maritieme projecten zoals baggeren, offshore-activiteiten en infrastructuurwerken. 
DEME is actief in het ontwerpen, bouwen en onderhouden van complexe civiele en waterbouwkundige systemen. Het bedrijf bezit meer dan 100 schepen die wereldwijd worden ingezet voor verschillende maritieme projecten.
Het probleem bevindt zich binnen de netwerkstructuur van een schip, die bestaat uit twee netwerken: 
het IT-netwerk voor administratieve taken, en het OT-netwerk voor de operationele systemen, zoals motoren, PLC's en sensoren. Deze twee netwerken zijn met elkaar verbonden via de nurse PC, die fungeert 
als toegangspunt tussen de netwerken. De focus van deze bachelorproef ligt op het afbakenen van deze kritieke schakel.

\subsection{Probleemstelling en hoofdonderzoeksvraag}
De ingenieurs van het ELEK-team aan boord moeten snel en efficiënt problemen kunnen oplossen via de nurse pc wanneer er storingen optreden in het OT- of IT-netwerk.
Als het probleem niet snel wordt verholpen, kunnen de werkzaamheden, zoals het vertrek van de boot of het baggeren, niet doorgaan, wat leidt tot aanzienlijke kosten. 
Er wordt vaak misbruik gemaakt van de nurse PC voor niet-geautoriseerde activiteiten (shadow IT), wat leidt tot verlies van controle over de veiligheid en toegang tot de nurse pc. 
Dit verhoogt de risico’s voor de netwerken aan boord. Het is daarom van cruciaal belang om de nurse pc goed af te bakenen en te beveiligen.
Op basis van deze probleemstelling wordt de hoofdonderzoeksvraag geformuleerd:
"Hoe kan de toegangsbeheer en de remote access van de nurse pc op een schip worden versterkt door een vulnerability assessment 
uit te voeren en vervolgens de juiste beveiligingsmaatregelen te implementeren om zo de veiligheid van zowel het IT- als OT-netwerk aan boord te garanderen?"

\subsection{Doelgroep}
De hoofdgroep van de doelgroep is het bedrijf DEME-group, aangezien een goed functionerende nurse pc en veilige netwerken 
essentieel zijn voor de continuïteit van kritieke operaties aan boord. Het oplossen van beveiligingsproblemen voorkomt financiële verliezen en verhoogt de efficiëntie van de schepen.
Binnen het bedrijf zijn drie specifieke sectoren die als doelgroepen worden beschouwd: het ELEK-team (elektroteam) aan boord bestaande uit meerdere ingenieurs elk met hun specialiteit, de IT- en netwerkbeheerders, en de scheepsoperators en leidinggevenden.
Het ELEK-team is is verantwoordelijk voor het onderhoud en de werking van de systemen aan boord. Wanneer er zich technische problemen voordoen, moeten zij snel en efficiënt kunnen troubleshooten met de nurse pc.
Een veilige nurse PC is essentieel voor hen om zonder vertraging toegang te krijgen tot de juiste systemen, zodat storingen in de IT- en OT-netwerken snel opgelost kunnen worden en het schip operationeel blijft.
De IT- en netwerkbeheerders moeten ervoor zorgen dat geen onbevoegden toegang krijgen en dat de systemen beschermd zijn tegen cyberdreigingen, wat de algehele netwerkveiligheid verbetert.
Scheepsoperators, kapiteins en leidinggevenden zijn verantwoordelijk voor de operationele veiligheid en efficiëntie aan boord. Een veilige nurse PC is cruciaal voor hen om storingen te voorkomen, zodat de schepen probleemloos kunnen opereren en de 
veiligheid van de bemanning en lading gewaarborgd blijft.

\subsection{Onderzoeksdoelstelling}
De doelstelling van dit onderzoek is om een vulnerability assessment uit te voeren van de nurse pc om kwetsbaarheden van het systeem te identificeren.
Op basis van de bevindingen worden de benodigde beveiligingsmaatregelen geïmplementeerd om de integriteit van de nurse pc te versterken.
Het eindresultaat van het onderzoek is een gedetailleerd rapport met de geïdentificeerde kwetsbaarheden, aanbevelingen voor verbeteringen en een proof-of-concept 
waarin de implementatie van de beveiligingsmaatregelen wordt gedemonstreerd.



%---------- Stand van zaken ---------------------------------------------------

\section{Literatuurstudie}%
\label{sec:literatuurstudie}

Hier beschrijf je de \emph{state-of-the-art} rondom je gekozen onderzoeksdomein, d.w.z.\ een inleidende, doorlopende tekst over het onderzoeksdomein van je bachelorproef. Je steunt daarbij heel sterk op de professionele \emph{vakliteratuur}, en niet zozeer op populariserende teksten voor een breed publiek. Wat is de huidige stand van zaken in dit domein, en wat zijn nog eventuele open vragen (die misschien de aanleiding waren tot je onderzoeksvraag!)?

Je mag de titel van deze sectie ook aanpassen (literatuurstudie, stand van zaken, enz.). Zijn er al gelijkaardige onderzoeken gevoerd? Wat concluderen ze? Wat is het verschil met jouw onderzoek?

Verwijs bij elke introductie van een term of bewering over het domein naar de vakliteratuur, bijvoorbeeld~\autocite{Hykes2013}! Denk zeker goed na welke werken je refereert en waarom.

Draag zorg voor correcte literatuurverwijzingen! Een bronvermelding hoort thuis \emph{binnen} de zin waar je je op die bron baseert, dus niet er buiten! Maak meteen een verwijzing als je gebruik maakt van een bron. Doe dit dus \emph{niet} aan het einde van een lange paragraaf. Baseer nooit teveel aansluitende tekst op eenzelfde bron.

Als je informatie over bronnen verzamelt in JabRef, zorg er dan voor dat alle nodige info aanwezig is om de bron terug te vinden (zoals uitvoerig besproken in de lessen Research Methods).

% Voor literatuurverwijzingen zijn er twee belangrijke commando's:
% \autocite{KEY} => (Auteur, jaartal) Gebruik dit als de naam van de auteur
%   geen onderdeel is van de zin.
% \textcite{KEY} => Auteur (jaartal)  Gebruik dit als de auteursnaam wel een
%   functie heeft in de zin (bv. ``Uit onderzoek door Doll & Hill (1954) bleek
%   ...'')

Je mag deze sectie nog verder onderverdelen in subsecties als dit de structuur van de tekst kan verduidelijken.

%---------- Methodologie ------------------------------------------------------
\section{Methodologie}%
\label{sec:methodologie}

Hier beschrijf je hoe je van plan bent het onderzoek te voeren. Welke onderzoekstechniek ga je toepassen om elk van je onderzoeksvragen te beantwoorden? Gebruik je hiervoor literatuurstudie, interviews met belanghebbenden (bv.~voor requirements-analyse), experimenten, simulaties, vergelijkende studie, risico-analyse, PoC, \ldots?

Valt je onderwerp onder één van de typische soorten bachelorproeven die besproken zijn in de lessen Research Methods (bv.\ vergelijkende studie of risico-analyse)? Zorg er dan ook voor dat we duidelijk de verschillende stappen terug vinden die we verwachten in dit soort onderzoek!

Vermijd onderzoekstechnieken die geen objectieve, meetbare resultaten kunnen opleveren. Enquêtes, bijvoorbeeld, zijn voor een bachelorproef informatica meestal \textbf{niet geschikt}. De antwoorden zijn eerder meningen dan feiten en in de praktijk blijkt het ook bijzonder moeilijk om voldoende respondenten te vinden. Studenten die een enquête willen voeren, hebben meestal ook geen goede definitie van de populatie, waardoor ook niet kan aangetoond worden dat eventuele resultaten representatief zijn.

Uit dit onderdeel moet duidelijk naar voor komen dat je bachelorproef ook technisch voldoen\-de diepgang zal bevatten. Het zou niet kloppen als een bachelorproef informatica ook door bv.\ een student marketing zou kunnen uitgevoerd worden.

Je beschrijft ook al welke tools (hardware, software, diensten, \ldots) je denkt hiervoor te gebruiken of te ontwikkelen.

Probeer ook een tijdschatting te maken. Hoe lang zal je met elke fase van je onderzoek bezig zijn en wat zijn de concrete \emph{deliverables} in elke fase?

%---------- Verwachte resultaten ----------------------------------------------
\section{Verwacht resultaat, conclusie}%
\label{sec:verwachte_resultaten}

Hier beschrijf je welke resultaten je verwacht. Als je metingen en simulaties uitvoert, kan je hier al mock-ups maken van de grafieken samen met de verwachte conclusies. Benoem zeker al je assen en de onderdelen van de grafiek die je gaat gebruiken. Dit zorgt ervoor dat je concreet weet welk soort data je moet verzamelen en hoe je die moet meten.

Wat heeft de doelgroep van je onderzoek aan het resultaat? Op welke manier zorgt jouw bachelorproef voor een meerwaarde?

Hier beschrijf je wat je verwacht uit je onderzoek, met de motivatie waarom. Het is \textbf{niet} erg indien uit je onderzoek andere resultaten en conclusies vloeien dan dat je hier beschrijft: het is dan juist interessant om te onderzoeken waarom jouw hypothesen niet overeenkomen met de resultaten.

